\documentclass{article}

\title{\textbf{Second laboratory work}}
\author{Evgenii Pishchik 206}

\usepackage{sagetex}
\setlength{\sagetexindent}{10ex}

\usepackage{geometry}
\geometry{a4paper, total={170mm,257mm}, left=20mm, right=20mm, top=20mm, bottom=20mm}

\usepackage{xcolor}
\usepackage{titlesec}

\definecolor{myred1}{RGB}{107, 14, 14}
\definecolor{mygreen1}{RGB}{0, 102, 0}

\begin{document}

\maketitle
\titleformat{\section}[block]{\color{red}\Large\bfseries\filcenter}{}{1em}{}

\section*{Task 3 variant 1}

\textbf{Create some constants which will be useful in future for plot 3d graphics.}
\begin{sageblock}
xmin, xmax = -10, 10
ymin, ymax = -10, 10
zmin, zmax = -10, 10
\end{sageblock}
\textbf{Say for sage that \textcolor{mygreen1}{x},  \textcolor{mygreen1}{y}, \textcolor{mygreen1}{z} is variables, and create function \textcolor{myred1}{f}.}
\begin{sageblock}
var("x y z")
f(x, y, z) = 7*x^2 + 3*y^2 + 3*z^2 +  8*x*y + 8*x*z + 6*y*z + 6*x + y + 7
\end{sageblock}
\textbf{Display the function formula.}

\textcolor{myred1}{$\sage{f(x, y, z)}$}\\
\textbf{Plot graphic of unchanged figure.} 
\begin{sageblock}
p = implicit_plot3d(f(x=x, y=y, z=z), (x, xmin, xmax), (y, ymin, ymax), (z, zmin, zmax))
\end{sageblock}

\begin{center}
\sageplot{p}
\end{center}
\textbf{Function that, according to a given algorithm, converts the function passed as an argument to the canonical form.
Return tuple (\textcolor{mygreen1}{lambdas}, \textcolor{mygreen1}{a}, \textcolor{mygreen1}{a0}).}
\begin{sageblock}
def kanonic_coeffs(fun):
    try:
        var("l l1 l2 l3")
        lvcts = []
        svcts = []
        tmp_fun = fun
        a = vector(RR, 3)
        tmp_vct = vector(RR, 9)
        var_combs_tmp = (x^2, y^2, z^2, x*y, x*z, y*z)
        var_combs_0 = (x^2, x*y, x*z, x*y, y^2, y*z, x*z, y*z, z^2)
        var_combs_1 = (x, y, z)
        for i, var_comb in enumerate(var_combs_0):
            if i == 0 or i == 4 or i == 8:
                tmp_vct[i] = fun.coefficient(var_comb)
            else:
                tmp_vct[i] = fun.coefficient(var_comb) / 2
        for var_comb in var_combs_tmp:
            tmp_fun -= fun.coefficient(var_comb)*var_comb
        for i, var_comb in enumerate(var_combs_1):
            a[i] = tmp_fun.coefficient(var_comb) / 2
            tmp_fun -= tmp_fun.coefficient(var_comb)*var_comb
        a0 = tmp_fun.n()    
        A = matrix(SR, 3, tmp_vct)
        L = matrix(SR, 3, 3, var('l'))
        E = matrix(SR, 3, 3, 1)
        lvct = vector([l1, l2, l3])
        lambdas = solve([(A-L).determinant() == 0], l)
        for i, el in enumerate(lambdas):
            nums = []
            lhs = (A-el.rhs()*E)*lvct
            res = solve([lhs[0] == 0, lhs[1] == 0, lhs[2] == 0], l1, l2, l3)[0]
            for i in range(len(res)):
                if len(res[i].rhs().variables()) == 0:
                    nums.append(res[i].rhs())
                else:
                    nums.append(res[i].rhs()(1))
            lvcts.append(vector(nums))
        for el in lvcts:
            norm_lvct = (el / sqrt((el*el).n())).n()
            svcts.append(norm_lvct)
        ST = matrix(RR, 3)
        for i in range(len(svcts)):
            ST[i] = svcts[i]
        a_ = ST*a
        for i in range(len(lambdas)):
            lambdas[i] = lambdas[i].rhs().n()
        return (lambdas, a_, a0)
    except:
        print("Something gone wrong\n")
        return (None, None, None)
\end{sageblock}
\textbf{Obtain from the function a tuple of the required coefficients.}
\begin{sageblock}
(lambdas, a, a0) = kanonic_coeffs(f)
\end{sageblock} 
\textbf{Display the obtained coefficients.}
\begin{sageblock}
print("Lambdas:", tuple(lambdas))
print("a:", a)
print("a0:", a0)
\end{sageblock} 

\textbf{\textcolor{mygreen1}{Lambdas:}} $\sage{tuple(lambdas)}$

\textbf{\textcolor{mygreen1}{a:}} $\sage{a}$

\textbf{\textcolor{mygreen1}{a0:}} $\sage{a0}$\\
\textbf{Create a function with obtained coefficients.}
\begin{sageblock}
var("kx ky kz")
first_part = lambdas[0]*kx^2 + lambdas[1]*ky^2 + lambdas[2]*kz^2
second_part = 2*a[0]*kx + 2*a[1]*ky + 2*a[2]*kz + a0
kanonic_func(kx, ky, kz) = first_part + second_part
\end{sageblock}
\textbf{Display the obtained function formula.}

\textcolor{myred1}{$\sage{kanonic_func(kx, ky, kz)}$}\\
\textbf{Plot graphic of changed figure.} 
\begin{sageblock}
p = implicit_plot3d(kanonic_func(kx=kx, ky=ky, kz=kz), (kx, xmin, xmax),\
(ky, ymin, ymax), (kz, zmin, zmax))
\end{sageblock}
\begin{center}
\sageplot{p}
\end{center}

\section*{Initial function.}
\begin{center}
\textcolor{myred1}{$7*x^2 + 3*y^2 + 3*z^2 +  8*x*y + 8*x*z + 6*y*z + 6*x + y + 7$}
\end{center}

\section*{Canonical function.}

\begin{center}
\textcolor{myred1}{$\sage{lambdas[0]}*kx^2+\sage{lambdas[1]}*ky^2 \sage{a[2]}*kz$}
\end{center}

\end{document}